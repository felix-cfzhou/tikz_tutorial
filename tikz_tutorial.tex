\documentclass[12pt]{article} 

\usepackage{fullpage}
\usepackage{bookmark}
\usepackage{amsmath}
\usepackage{amssymb}
\usepackage[dvipsnames]{xcolor}
\usepackage{hyperref} % for the URL
\usepackage[shortlabels]{enumitem}
\usepackage{mathtools}
\usepackage[most]{tcolorbox}
\usepackage[amsmath,standard,thmmarks]{ntheorem} 
\usepackage{physics}
\usepackage{pst-tree} % for the trees
\usepackage{verbatim} % for comments, for version control
\usepackage{tabu}
\usepackage{tikz}
\usepackage{float}

\lstnewenvironment{python}{
\lstset{frame=tb,
language=Python,
aboveskip=3mm,
belowskip=3mm,
showstringspaces=false,
columns=flexible,
basicstyle={\small\ttfamily},
numbers=none,
numberstyle=\tiny\color{Green},
keywordstyle=\color{Violet},
commentstyle=\color{Gray},
stringstyle=\color{Brown},
breaklines=true,
breakatwhitespace=true,
tabsize=2}
}
{}

\lstnewenvironment{latex}{
\lstset{
backgroundcolor=\color{white!90!NavyBlue},   % choose the background color; you must add \usepackage{color} or \usepackage{xcolor}; should come as last argument
basicstyle={\scriptsize\ttfamily},        % the size of the fonts that are used for the code
breakatwhitespace=false,         % sets if automatic breaks should only happen at whitespace
breaklines=true,                 % sets automatic line breaking
captionpos=b,                    % sets the caption-position to bottom
commentstyle=\color{Gray},    % comment style
deletekeywords={...},            % if you want to delete keywords from the given language
escapeinside={\%*}{*)},          % if you want to add LaTeX within your code
extendedchars=true,              % lets you use non-ASCII characters; for 8-bits encodings only, does not work with UTF-8
% firstnumber=1000,                % start line enumeration with line 1000
frame=single,	                   % adds a frame around the code
keepspaces=true,                 % keeps spaces in text, useful for keeping indentation of code (possibly needs columns=flexible)
keywordstyle=\color{Violet},       % keyword style
language=[latex]tex,                 % the language of the code
morekeywords={*,...},            % if you want to add more keywords to the set
% numbers=left,                    % where to put the line-numbers; possible values are (none, left, right)
% numbersep=5pt,                   % how far the line-numbers are from the code
% numberstyle=\tiny\color{Green}, % the style that is used for the line-numbers
rulecolor=\color{black},         % if not set, the frame-color may be changed on line-breaks within not-black text (e.g. comments (green here))
showspaces=false,                % show spaces everywhere adding particular underscores; it overrides 'showstringspaces'
showstringspaces=false,          % underline spaces within strings only
showtabs=false,                  % show tabs within strings adding particular underscores
stepnumber=2,                    % the step between two line-numbers. If it's 1, each line will be numbered
stringstyle=\color{Brown},     % string literal style
tabsize=2,	                   % sets default tabsize to 2 spaces
title=\lstname}                   % show the filename of files included with \lstinputlisting; also try caption instead of title
}
{}

% floor, ceiling, set
\DeclarePairedDelimiter{\ceil}{\lceil}{\rceil}
\DeclarePairedDelimiter{\floor}{\lfloor}{\rfloor}
\DeclarePairedDelimiter{\set}{\lbrace}{\rbrace}
\DeclarePairedDelimiter{\iprod}{\langle}{\rangle}

\DeclareMathOperator{\Int}{int}
\DeclareMathOperator{\mean}{mean}

% commonly used sets
\newcommand{\R}{\mathbb{R}}
\newcommand{\N}{\mathbb{N}}
\newcommand{\Q}{\mathbb{Q}}
\renewcommand{\P}{\mathbb{P}}

\newcommand{\sset}{\subseteq}

\theoremstyle{break}
\theorembodyfont{\upshape}

\newtheorem{thm}{Theorem}[subsection]
\tcolorboxenvironment{thm}{
enhanced jigsaw,
colframe=Dandelion,
colback=White!90!Dandelion,
drop fuzzy shadow east,
rightrule=2mm,
sharp corners,
before skip=10pt,after skip=10pt
}

\newtheorem{cor}{Corollary}[thm]
\tcolorboxenvironment{cor}{
boxrule=0pt,
boxsep=0pt,
colback={White!90!RoyalPurple},
enhanced jigsaw,
borderline west={2pt}{0pt}{RoyalPurple},
sharp corners,
before skip=10pt,
after skip=10pt,
breakable
}

\newtheorem{lem}[thm]{Lemma}
\tcolorboxenvironment{lem}{
enhanced jigsaw,
colframe=Red,
colback={White!95!Red},
rightrule=2mm,
sharp corners,
before skip=10pt,after skip=10pt
}

\newtheorem{ex}[thm]{Example}
\tcolorboxenvironment{ex}{% from ntheorem
blanker,left=5mm,
sharp corners,
before skip=10pt,after skip=10pt,
borderline west={2pt}{0pt}{Gray}
}

\newtheorem*{pf}{Proof}
\tcolorboxenvironment{pf}{% from ntheorem
breakable,blanker,left=5mm,
sharp corners,
before skip=10pt,after skip=10pt,
borderline west={2pt}{0pt}{NavyBlue!80!white}
}

\newtheorem{defn}{Definition}[subsection]
\tcolorboxenvironment{defn}{
enhanced jigsaw,
colframe=Cerulean,
colback=White!90!Cerulean,
drop fuzzy shadow east,
rightrule=2mm,
sharp corners,
before skip=10pt,after skip=10pt
}

\newtheorem{prop}[thm]{Proposition}
\tcolorboxenvironment{prop}{
boxrule=0pt,
boxsep=0pt,
colback={White!90!Green},
enhanced jigsaw,
borderline west={2pt}{0pt}{Green},
sharp corners,
before skip=10pt,
after skip=10pt,
breakable
}

\setlength\parindent{0pt}
\setlength{\parskip}{2pt}


\begin{document}
\let\ref\Cref

\title{\bf{A Tiktz Tutorial}}
\date{\today}
\author{Felix Zhou}

\maketitle
\newpage
\tableofcontents
\listoffigures
\listoftables
\newpage

\section*{Introduction}
\subsection*{What it this?}
I am an upcoming third year (3A) student at the University of Waterloo who plans to take CO342 - Introduction to Graph Theory, in Fall 2019.
To aid me and other students such as myself with typesetting graphs using tikz, I created this shortened tutorial from the TikZ and PGF Manual.
Please do not hesitate to reach out to me if there are any errors and I will publish an updated version ASAP.

\subsection*{Goal}
I have always believed in understanding your tools completely before using them.
In addition to my tutorial on TikZ, I will include as much notes on the math behind non-trivial TikZ commands as I can, with the condition that I myself at least understand the gist of the technologies.
If it is completey above me and looks like gobly goop, I will clearly note this and invite \textit{you} to write me a short section based on your understanding.


\newpage
\section{Generic Setup}
\begin{latex}
    \documentclass{article}
    \usepackage{tikz}
    \usepackage{float}

    \begin{document}
    \begin{figure}[H] % forces position of tikz to be where it is in source file
        \centering

        \begin{tikzpicture}
            % insert provided code
        \end{tikzpicture}

        \caption{A Caption}
        \label{fig:alabel}
    \end{figure}
    \end{document}
\end{latex}

This is our default setup for any document classes involving TikZ.
If additional imports are necessary, we will explicityly note them.

Note that we may ommit typing ``tikzpicture'' everytime and inline simple TikZ pictures with the ``tikz'' command.
When we do this, we will explicitly type out ``tikz``.


\newpage
\section{Paths}

\subsection{Straight Paths}
\begin{figure}[H]
    \centering
    \begin{tikzpicture}
        \draw (-1.5,0) -- (1.5,0);
        \draw (0,-1.5) -- (0,1.5);
    \end{tikzpicture}.
    \caption{Basic Straight Path}
    \label{fig:basicstraightpath}
\end{figure}

\begin{latex}
    \draw (-1.5,0) -- (1.5,0); % draw straight lines between two points
    \draw (0,-1.5) -- (0,1.5);
\end{latex}

We can also directly inline a similar picture
\begin{figure}[H]
    \centering
    \tikz \draw (-1.5,0) -- (1.5,0) -- (0,-1.5) -- (0,1.5);
    \caption{Inlined Straight Path}
    \label{fig:inlinedstraightpath}
\end{figure}

\begin{latex}
    \tikz \draw (-1.5,0) -- (1.5,0) -- (0,-1.5) -- (0,1.5);
\end{latex}

\subsection{Curved Paths}
TikZ allows us to define arbitrary curves using control points.
The basis behind of this concept are B\'ezier Curves, which in turn are based on the Bernstein Polynomial.

\begin{figure}[H]
    \centering
    \begin{tikzpicture}
        \filldraw [gray]
        (0,0) circle (2pt)
        (1,1) circle (2pt)
        (3,3) circle (2pt)
        (3,0) circle (2pt);
        \draw (0,0) .. controls (1,1) and (3,3) .. (3,0);
    \end{tikzpicture}
    \caption{Basic Curve}
    \label{fig:basiccurve}
\end{figure}

\begin{latex}
    \filldraw [gray]
    (0,0) circle (2pt)
    (1,1) circle (2pt)
    (3,3) circle (2pt)
    (3,0) circle (2pt);
    \draw (0,0) .. controls (1,1) and (3,3) .. (3,0);
\end{latex}

The above shows a basic curve where the control points are explicitly drawn as well.
Obviously, this would be ommited (on your graph theory course home work for example).

\subsubsection{Bezier Curves and the Bernstein Polynomial}
This section gives some mathematical background on curved paths and is skippable.

\begin{defn}[B\'ezier Curve]
   A recursive definition for the B\'ezier curve of degree $n$ expresses it as a point-to-point linear combination (interpolation) of a pair of corresponding points in two B\'ezier curves of degree $n-1$.

   \begin{align*}
       B &: \R \to \R \\
       B_{P_0}(t) &= P_0 &&\text{base case} \\
       B(t) &= B_{P_0P_1\dots P_n}(t) = (1-t)B_{P_0P_1\dots P_{n-1}}(t) + tB_{P_1P_2\dots P_n}(t)
   \end{align*}
\end{defn}

\begin{defn}[Bernstein Basis Polynomials]
    of degree $n$ are given by
    \[
    \set*{b_{i, n}(t) = \binom{n}{i} t^i (1-t)^{n-i} : 0\leq i\leq n}
    \]

    which form a basis of polynomials with degree at most $n$.
\end{defn}

\begin{prop}
    \[
    B(t) = \sum_{i=0}^n \binom{n}{i} t^i(1-t)^{n-i} P_i
    \]

    So we can express a B\'ezier curve as a linear combination of the Berstein Basis.
\end{prop}

The points $P_i$ are called \textit{Control Points} for the B\'ezier Curve.
The polygon formed by connecting the B\'ezier points with lines, starting with $P_0$ and finishing with $P_n$, gives the \textit{B\'ezier Polygon (Control Polygon)}.
The convex hull of the B\'ezier Polygon contains the B\'ezier Curve.

\subsection{Circular Paths}
Although it would certainly be possible to draw all our circular paths with control points, it would prove tedious to say the least.

\begin{figure}[H]
    \centering
    \tikz \draw (0,0) circle (30pt);
    \caption{Basic Circle}
    \label{fig:basiccircle}
\end{figure}

\begin{latex}
    \tikz \draw (0,0) circle (30pt);
\end{latex}

We can also draw Ellipses.

\begin{figure}[H]
    \centering
    \tikz \draw (0,0) ellipse (20pt and 10pt);
    \caption{Basic Ellipse}
    \label{fig:basicellipse}
\end{figure}

\begin{latex}
    \tikz \draw (0,0) ellipse (20pt and 10pt);
\end{latex}

Although it is possible to draw ellipses which are rotated in arbitrary directions, we will leave this for the section on transformations later on.

\end{document}


